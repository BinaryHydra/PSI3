\documentclass{VUMIFPSkursinis}
\usepackage{algorithmicx}
\usepackage{algorithm}
\usepackage{algpseudocode}
\usepackage{amsfonts}
\usepackage{amsmath}
\usepackage{booktabs}
\usepackage{blindtext}
\usepackage{bm}
\usepackage{caption}
\usepackage{color}
\usepackage{colortbl}
\usepackage{float}
\usepackage{graphicx}
\usepackage{listings}
\usepackage{multirow}
\usepackage{scrextend}
\usepackage{subfig}
\usepackage{wrapfig}
\usepackage{longtable}
\usepackage{enumitem}
\usepackage{xparse}
%\usepackage{tabularx}
\usepackage{ltxtable}
\usepackage{tabu}
\usepackage{xcolor}

% Titulinio aprašas
\university{Vilniaus universitetas}
\faculty{Matematikos ir informatikos fakultetas}
\department{Programų sistemų katedra}
\papertype{1 laboratorinis darbas}
\title{Grupinių Sporto Susitikimų Platforma}
\titleineng{The Platform for Group Sport Events}
\status{2 kurso 5 grupės studentai}
\author{Margiris Strakšys}
\secondauthor{Gabrielė Saletytė}
\thirdauthor{Vytautas Strimaitis}
\fourthauthor{Gabijus Arūnas Šukaitis}
\supervisor{lekt. Vytautas Valaitis}
\date{Vilnius – \the\year}

% Nustatymai
% \setmainfont{Palemonas}   % Pakeisti teksto šriftą į Palemonas (turi būti įdiegtas sistemoje)
%\bibliography{bibliografija}
\input{customCommands}

%========================================================================== %Dokumento pradžia %========================================================================
\begin{document}
    \pagenumbering{gobble}
    \maketitle
    \tableofcontents
	\pagenumbering{arabic}
    
    \sectionnonum{Anotacija} \label{anotacija}
		Šiame laboratoriniame darbe bus apibrėžti reikalavimai, struktūrinis srities modelis bei užduotys sistemai. Visas darbas paremtas ICONIX proceso metodologija.
		Darbas parengtas atsižvelgiant į pradinius užsakovo pateiktus reikalavimus.
    \sectionnonum{Įvadas} \label{ivadas}
        \subsection*{Programų sistemos pavadinimas} \label{ivadas_psPavadinimas}
            ,,Grupinių Sporto Susitikimų Platforma'' (sutrumpintas sistemos pavadinimas - ,,GSSP'').
        \subsection*{Darbo tikslas} \label{ivadas_darboTikslas}
            Remiantis ICONIX proceso principais, apibrėžti reikalavimus, struktūrinį srities modelį bei užduotis būsimai sistemai.
        \subsection*{Temos aktualumas} \label{ivadas_aktualumas}
            Projektas skirtas populiarinti sportą, gerinti sporto susitikimų organizuotumą. Kadangi sporto susitikimų sritis yra mažai 
            išvystyta, nors nemažai žmonių tokius susitikimus norėtų organizuoti, bus kuriama sistema, palengvinanti tokio tipo susitikimų 
            organizavimą bei dalyvavimą juose.
        \subsection*{Naudotojai} \label{ivadas_naudotojai}
            Sistema skirta sistemos administratoriui, klientams (lankytojams) bei rėmėjams.
        \subsection*{Darbo pagrindas} \label{ivadas_pagrindas}
            Dokumentas parengtas kaip programų sistemų inžinerijos laboratorinis darbas.
    \section{Reikalavimai} \label{reikalavimai}
		Šiame skyriuje aptariami programų sistemos funkciniai ir nefunkciniai reikalavimai. Jie sudaryti remiantis iš užsakovo gautais pradiniais
		reikalavimais, kurie pateikti šio laboratorinio darbo priede.
        \subsection{Funkciniai reikalavimai} \label{reikalavimai_fr}
			\begin{enumerate}[label=\textbf{FR\arabic*}]
				\item Svetainė turi būti pasiekiama:
					\begin{enumerate}[label*=\textbf{.\arabic*}]
						\item Neužsiregistravusiems vartotojams.
						\item Užsiregistravusiems vartotojams.
					\end{enumerate}
				\item Neprisijungęs/Neužsiregistravęs vartotojas užėjęs į puslapį gali pasiekti:
					\begin{enumerate}[label*=\textbf{.\arabic*}]
						\item Naujienas.
						\item Renginių kalendorius.
						\item Vaizdo įrašai.
						\item Pateikti pasiūlymą.
					\end{enumerate}	
				\item Prisijungęs vartotojas gali:
					\begin{enumerate}[label*=\textbf{.\arabic*}]
						\item Pirkti biletus.
						\item Užsiregistruoti į renginį.
						\item Aplikuoti į siūlomas darbo vietas.
					\end{enumerate}
				\item Registruodamas vartotojas pateikia:
					\begin{enumerate}[label*=\textbf{.\arabic*}]
						\item Vartotojo vardą.(būtina)
						\item Slaptažodį.(būtina)
						\item Vardą. (neprivaloma)
						\item Pavardę. (neprivaloma)
						\item Gimimo datą. (neprivaloma)
						\item Telefono numerį. (neprivaloma)
						\item Gyvenamąją vietą. (neprivaloma)
					\end{enumerate}
				\item Registruojantis į renginį arba aplikuojant į darbo pozicijas privalomą užpildyti visus papildomos informacijos anketos laukus.
				\item Registruojantis į komandinį renginį reikia pasirinkti arba sukurti komandą, su kuria bus dalyvaujamą.
				\item Kad sukurti komandą reikia nurodyti:
					\begin{enumerate}[label*=\textbf{.\arabic*}]
						\item Komandos pavadinimą.
						\item Žmones, kuriems bus išsiusti kvietimai prisijungti prie komandos.
					\end{enumerate}
				\item Vartotojas gavęs pakvietimą į komandą gali jį priimti arba atmesti.
				\item Tiesioginiai rodomus renginius galima rikiuoti pagal:
					\begin{enumerate}[label*=\textbf{.\arabic*}]
						\item Pradžios laiką.
						\item Peržiūrų skaičių.
					\end{enumerate}
				\item Prie kiekvieno renginio turi būti pateikta rezultatų lentelė.
				\item Dalyviai rezultatuose rikiuojami pagal savo pasiekta rezultatą.
				\item Rezultatų lentelę turi būti įmanoma filtruoti pagal visas pateiktas reikšmes.
                \item Sistema turi saugoti ir pateikti rezultatus ir bendrą statistiką:
					\begin{enumerate}[label*=\textbf{.\arabic*}]
						\item Sporto šakų.
						\item Invidualių dalyvių.
						\item Komandų.
					\end{enumerate}
				\item Skiltis “Pateikti pasiūlymą” leidžia pateikti pasiūlymą renginio organizatoriams.
				\item Pateikiant pasiūlymą reikia užpildyti:
					\begin{enumerate}[label*=\textbf{.\arabic*}]
						\item Savo elektroninį paštą.
						    \begin{enumerate}[label*=\textbf{.\arabic*}]
						        \item Prisijungusiam vartotojui užsipildo automatiškai.
						    \end{enumerate}
						\item Trumpą idėjos aprašymą.
					\end{enumerate}
				\item Svetainės rėmėjų logotipai pateikti puslapio viršuje, kairėje, dešinėje ir apačioje.
				\item Administratorius prisijungęs prie svetainės turi prieigą prie administratoriaus skilties.
				\item Per administratoriaus skiltį galima:
					\begin{enumerate}[label*=\textbf{.\arabic*}]
						\item Naujienas, renginius, renginių rezultatas, vaizdo įrašus:
						    \begin{enumerate}[label*=\textbf{.\arabic*}]
						        \item Peržiūrėti.
						        \item Sukurti.
						        \item Atnaujinti.
						        \item Ištrinti.
						    \end{enumerate}
						\item Trumpą idėjos aprašymą:
                            \begin{enumerate}[label*=\textbf{.\arabic*}]
						        \item Apriboti.
						        \item Nutraukti.
						    \end{enumerate}
						\item Peržiūrėti visų dalyvių, renginių, darbo aplikacijų sąrašus.
					\end{enumerate}
			    \item Prie kiekvieno renginio turi būti bilietų skiltis, kur turi būti nurodytą:
					\begin{enumerate}[label*=\textbf{.\arabic*}]
						\item Vieno bileto kainą.
						\item Likusiu bilietų kiekis.
					\end{enumerate}
				\item Internetinis puslapis turi būti pateiktas anglų, rusų bei lietuvių kalba.
				\item  Internetinis puslapis turi buti pasiekiamas ir naudojant mobilųjį įrenginį.
			\end{enumerate}

    \section{Struktūrinis dalykinės srities modelis} \label{strukturinisDSModelis}
		Šiame skyriuje konstruojamas struktūrinis dalykinės srities modelis, t.y. apibrėžiamos pagrindinės su sistema susijusios esybės, nurodomi ryšiai tarp jų.
        \subsection{Esybių diagrama} \label{strukturinisDSModelis_esybiuDiagrama}
            \begin{figure}[H]
                \centering
                \includegraphics[width=\textwidth, height=20cm, keepaspectratio]{img/MLP.png}
                \caption{Dalykinės srities modelio UML klasių diagrama}
                \label{fig:DS-klasiu-diagrama}
            \end{figure}
        \subsection*{Žodynas} \label{strukturinisDSModelis_zodynas}
            \begin{enumerate}[label=\textbf{E\arabic*.}]
                \item \textbf{Renginys} - ,,GSSP'' sistemoje esantis sporto renginys.
				\item \textbf{Privatus renginys} - „GSSP“ sistemoje esantis sporto renginys, skirtas tik tam tikrai grupei neįtraukiant pašalinių asmenų.
				\item \textbf{Viešas renginys} - „GSSP“ sistemoje esantis sporto renginys, skirtas visiems, norintiems dalyvauti.
                \item \textbf{Organizatorius} - asmuo, turintis teisę kurti ir valdyti savo sukurtus renginius.
                \item \textbf{Vartotojas} - asmuo, besinaudojantis sistema ir turintis galimybę prisijungti prie 
				esamo renginio.
				\item \textbf{Prisijungęs vartotojas} - vartotojas, turintis paskyrą „GSSP“ sistemoje.
				\item \textbf{Neprisijungęs vartotojas} - vartotojas, neturintis „GSSP“ paskyros.
				\item \textbf{Privilegijuotas vartotojas} - prisijungęs vartotojas, įgijęs privilegijuoto vartotojo sąsają.
                \item \textbf{Žemėlapis} - vietos keitimo procese naudojamas žemėlapis.
				\item \textbf{Sporto šaka} - sportinė veiklos sritis, kuri turi savo taisykles ir inventorių.
				\item \textbf{Renginių tvarkaraštis} - sistemoje esantis tvarkaraštis su išrikiuotu renginių sąrašu pagal datą.
				\item \textbf{Renginių sąrašas} - sąrašas, kuriame yra visi „GSSP“ sporto renginiai.
				\item \textbf{Atsakingas asmuo} - organizatorius arba organizatoriaus paskirtas asmuo, atsakingas už tam tikrą sporto renginį.
				\item \textbf{Reitingas} - vartotojo statusas, pagal kurį nustatoma, ar vartotojas gali dalyvauti renginyje.
				\item \textbf{Kalba} - vartotojo pasirinkta kalba, kuria pateikiama visa sistemoje esanti informacija.
				\item \textbf{Tema} - vartotojo pasirinkta tema (šviesi arba tamsi), kuria vaizduojama visa sistemos pateikiama informacija.
				\item \textbf{Reklama} - rėmėjų pateikta vaizdinė reklama, rodoma visiems, išskyrus privilegijuotus vartotojus.
				\item \textbf{Atsiliepimas} - raštiškas vartotojo atsakas organizatoriui apie jau įvykusį renginį.
				\item \textbf{Grafinė sąsaja} - vaizdinė „GSSP“ sąsaja, leidžianti vartotojams patogiai naudotis sistema.
				\item \textbf{Prisijungusio vartotojo sąsaja} - vaizdinė „GSSP“ sąsaja, leidžianti vartotojams patogiai naudotis sistema, turinti papildomų galimybių.
				\item \textbf{Paprasto vartotojo sąsaja} - prisijungusio vartotojo „GSSP“ sąsaja, neturinti privilegijuoto statuso.
				\item \textbf{Privilegijuoto vartotojo sąsaja} - prisijungusio vartotojo „GSSP“ sąsaja, kurioje nėra vaizduojamos reklamos.
				\item \textbf{Neprisijungusio vartotojo sąsaja} - įprasta sąsaja, kurioje leidžiamos tik bendros prieigos.
				
            \end{enumerate}
        \subsection{Reikalavimų - struktūrinio modelio atsekamumo matrica}\label{strukturinisDSModelis_matrica}
			\begin{table}[H]
				\centering
				\caption{Reikalavimų - struktūrinio modelio atsekamumo matrica}
				\label{ReikalavimųStruktūrinioModelioAtsekamumoMatrica}
				\begin{tabular}{|
				>{\columncolor[HTML]{9B9B9B}}l |l|l|l|l|l|l|l|l|l|l|l|l|l|l|l|}
				\hline 
				X   & \cellcolor[HTML]{9B9B9B}\rotatebox[origin=c]{90}{FR1} & \cellcolor[HTML]{9B9B9B}\rotatebox[origin=c]{90}{FR2} & \cellcolor[HTML]{9B9B9B}\rotatebox[origin=c]{90}{FR3}& \cellcolor[HTML]{9B9B9B}\rotatebox[origin=c]{90}{FR4} & \cellcolor[HTML]{9B9B9B}\rotatebox[origin=c]{90}{FR5} & \cellcolor[HTML]{9B9B9B}\rotatebox[origin=c]{90}{FR6}& \cellcolor[HTML]{9B9B9B}\rotatebox[origin=c]{90}{FR7} & \cellcolor[HTML]{9B9B9B}\rotatebox[origin=c]{90}{FR8} & \cellcolor[HTML]{9B9B9B}\rotatebox[origin=c]{90}{NFR1} & \cellcolor[HTML]{9B9B9B}\rotatebox[origin=c]{90}{NFR2} & \cellcolor[HTML]{9B9B9B}\rotatebox[origin=c]{90}{NFR3} & \cellcolor[HTML]{9B9B9B}\rotatebox[origin=c]{90}{NFR4} & \cellcolor[HTML]{9B9B9B}\rotatebox[origin=c]{90}{NFR5} & \cellcolor[HTML]{9B9B9B}\rotatebox[origin=c]{90}{NFR6} & \cellcolor[HTML]{9B9B9B}\rotatebox[origin=c]{90}{NFR7} \\ \hline
				E1  & X                           & X                           &                             & X                           & X                           & X                           & X                           & X                           & X                            & X                            &                              &                              &                              & X                            &                              \\ \hline
				E2  & X                           &                             &                             & X                           & X                           & X                           & X                           & X                           &                              & X                            &                              &                              &                              & X                            &                              \\ \hline
				E3  & X                           & X                           &                             & X                           & X                           & X                           & X                           & X                           & X                            & X                            &                              &                              &                              & X                            &                              \\ \hline
				E4  &                             & X                           &                             & X                           & X                           &                             & X                           &                             &                              &                              & X                            &                              &                              & X                            &                              \\ \hline
				E5  &                             & X                           & X                           & X                           & X                           &                             & X                           & X                           &                              & X                            & X                            &                              &                              &                              & X                            \\ \hline
				E6  &                             & X                           & X                           & X                           & X                           &                             & X                           & X                           &                              &                              &                              &                              &                              &                              & X                            \\ \hline
				E7  &                             & X                           & X                           &                             & X                           &                             &                             &                             &                              &                              &                              &                              &                              &                              &                              \\ \hline
				E8  &                             & X                           &                             & X                           & X                           &                             & X                           & X                           &                              & X                            & X                            &                              &                              &                              & X                            \\ \hline
				E9  &                             &                             &                             &                             &                             &                             &                             &                             & X                            &                              &                              &                              &                              &                              &                              \\ \hline
				E10 &                             &                             &                             &                             &                             & X                           &                             &                             &                              &                              &                              &                              &                              &                              &                              \\ \hline
				E11 &                             & X                           &                             & X                           &                             &                             &                             &                             & X                            &                              &                              &                              &                              &                              &                              \\ \hline
				E12 &                             & X                           &                             & X                           &                             & X                           & X                           &                             & X                            &                              &                              &                              &                              &                              &                              \\ \hline
				E13 &                             &                             &                             &                             &                             &                             & X                           &                             &                              &                              &                              &                              &                              &                              &                              \\ \hline
				E14 &                             &                             &                             &                             &                             &                             &                             & X                           &                              &                              &                              &                              &                              &                              & X                            \\ \hline
				E15 &                             &                             &                             &                             &                             &                             &                             &                             &                              &                              &                              &                              & X                            &                              &                              \\ \hline
				E16 &                             &                             &                             &                             &                             &                             &                             &                             &                              &                              &                              & X                            &                              &                              &                              \\ \hline
				E17 &                             &                             &                             &                             &                             &                             &                             &                             &                              & X                            &                              &                              &                              &                              &                              \\ \hline
				E18 &                             &                             &                             & X                           &                             &                             &                             &                             &                              &                              &                              &                              &                              &                              &                              \\ \hline
				E19 &                             &                             &                             &                             &                             &                             &                             &                             &                              &                              &                              & X                            & X                            &                              &                              \\ \hline
				E20 &                             & X                           & X                           & X                           & X                           &                             &                             &                             &                              & X                            &                              &                              &                              &                              & X                            \\ \hline
				E21 &                             & X                           &                             & X                           & X                           &                             &                             &                             &                              & X                            &                              &                              &                              &                              & X                            \\ \hline
				E22 &                             & X                           &                             &                             & X                           &                             &                             &                             &                              &                              &                              &                              &                              &                              & X                            \\ \hline
				E23 &                             & X                           & X                           &                             & X                           &                             &                             &                             &                              & X                            &                              &                              &                              &                              &                              \\ \hline
				\end{tabular}
			\end{table}    
    \section{Užduotys}\label{uzduotys}
		Šiame skyriuje pateikiamos pagrindinės neprisijungusio, prisijungusio vartotojo, organizatoriaus ir atsakingo už renginį asmens užduotys.
		Taip pat pateikiama bendra UML diagrama apibrėžianti visų šių agentų užduotis.
			\noindent
			\begin{figure}[H]
                \centering
                \includegraphics[width=\textwidth]{img/MLP.png}
                \caption{UML užduočių diagrama}
                \label{fig:uzduociu-diagrama}
            \end{figure}

		\begin{enumerate} [label = \textbf{U\arabic*.}]
			\item \textbf{Užsiregistruoti} \\
				Vartotojas spaudžia mygtuką ,,Užsiregistruoti''.
				Vartotojas įveda informaciją apie save (Vardas, pavardė, prisijungimo vardas, slaptažodis).
				Vartotojas spaudžia mygtuką ,,Patvirtinti registraciją''. 
				Sistema užregistruoja nauja vartotoją. Sistema atidaro naują langą su užrašu ,,Sėkminga registracija''.
				
				\underline{Alternatyvūs scenarijai:}
				\begin{itemize}
					\item Jeigu vartotojo informacija įvesta neteisingai arba kuris nors registracijos laukelis nėra užpildytas, 
					sistema neleidžia paspausti mygtuko ,,Patvirtinti registraciją''. 
				\end{itemize}

				\begin{figure}[H]
					\centering
					\includegraphics[width=\textwidth, height=11cm, keepaspectratio]{img/MLP.png}
					\caption{Registracijos langas}
					\label{fig:uzd_registracija}
				\end{figure}
			\item \textbf{Prisijungti} \\
				Vartotojas spaudžia mygtuką ,,Prisijungti''. 
				Vartotojas įveda prisijungimo vardą ir slaptažodį.	
				Sistema patikrina, ar įvesti duomenys yra teisingi. 
				Sistema vartotojui atidaro prisijungusio vartotojo grafinę sąsają. 
				Vartotojas gali naudoti prisijungusio vartotojo funkcijas.
				
				\underline{Alternatyvūs scenarijai:}
				\begin{itemize}
					\item Jeigu vartotojas įveda neteisingą prisijungimo informaciją ar neužpildo kurio nors prisijungimo laukelio, 
					sistema paprašo dar kartą įvesti duomenis.
				\end{itemize}

				\begin{figure}[H]
						\centering
						\includegraphics[width=\textwidth, height=5cm, keepaspectratio]{img/MLP.png}
						\caption{Prisijungimo langas}
						\label{fig:prisijungimo-langas}
					\end{figure}
				\item \textbf{Peržiūrėti ateityje vyksiančių renginių sąrašą} \\
					Vartotojas spaudžia mygtuką ,,Renginiai''. 
					Atsivėrusiame pasirinkimų lange vartotojas pasirenka ,,Renginių sąrašas''.
					Sistema atidaro viešų renginių, kurie vyks nuo tos akimirkos (minutės tikslumu), sąrašą. 
					Sistema rodo 15 artimiausių renginių viename puslapyje.
					
					\underline{Alternatyvūs scenarijai:}
					\begin{itemize}
						\item Jeigu sistemoje yra mažiau nei 15 renginių, pirmame puslapyje vaizduojami visi esantys renginiai. 
						\item Jeigu sistemoje nėra renginių, vartotojui pateikiama žinutė:
						,,Atsiprašome, jums prieinamų renginių nėra. Norėdami pamatyti daugiau pasirinkimų, prisijunkite.''
					\end{itemize}

				\begin{figure}[H]
					\centering
					\includegraphics[width=\textwidth]{img/MLP.png}
					\caption{Renginių sąrašo langas}
					\label{fig:uzd_perziureti-vyksianciu-renginiu-sarasa}
				\end{figure}
			\newpage
			\item \textbf{Peržiūrėti preliminarų renginių tvarkaraštį} \\
				Vartotojas spaudžia mygtuką ,,Renginiai''. 
				Atsivėrusiame pasirinkimų lange vartotojas pasirenka ,,Renginių sąrašas''.
				Sistema atidaro visų vykstančių ir vyksiančių viešų renginių tvarkaraštį. 
				Sistema rodo 15 artimiausių renginių per vieną puslapį.
				
				\underline{Alternatyvūs scenarijai:}
				\begin{itemize}
					\item Jeigu sistemoje yra mažiau nei 15 renginių, pirmame puslapyje vaizduojami visi esantys renginiai. 
					\item Jeigu sistemoje nėra renginių, vartotojui pateikiama žinutė:
					,,Atsiprašome, jums prieinamų renginių nėra. Norėdami pamatyti daugiau pasirinkimų, prisijunkite.''
				\end{itemize}

				\begin{figure}[H]
					\centering
					\includegraphics[width=\textwidth, height=8.5cm, keepaspectratio]{img/MLP.png}
					\caption{Preliminaraus renginių tvarkaraščio langas}
					\label{fig:prelim-tvark-langas}
				\end{figure}
			\item \textbf{Peržiūrėti preliminarų renginių tvarkaraštį} \\
					Vartotojas spaudžia mygtuką ,,Renginiai''. 
					Atsivėrusiame pasirinkimų lange vartotojas pasirenka ,,Renginių sąrašas''.
					Sistema atidaro visų vykstančių ir vyksiančių viešų renginių tvarkaraštį. 
					Sistema rodo 15 artimiausių renginių viename puslapyje.
					
					\underline{Alternatyvūs scenarijai:}
					\begin{itemize}
						\item Jeigu sistemoje yra mažiau nei 15 renginių, pirmame puslapyje vaizduojami visi esantys renginiai. 
						\item Jeigu sistemoje nėra renginių, vartotojui pateikiama žinutė:
						,,Atsiprašome, jums prieinamų renginių nėra. Norėdami pamatyti daugiau pasirinkimų, prisijunkite.''
					\end{itemize}

				\begin{figure}[H]
					\centering
					\includegraphics[width=\textwidth]{img/MLP.png}
					\caption{Konkretaus renginio langas}
					\label{fig:konkretaus-renginio-langas}
				\end{figure}
			\item \textbf{Prisijungti prie viešo renginio} \\
					Vartotojas spaudžia mygtuką ,,Renginiai''. 
					Vartotojas pasirenka konkretų renginį paspausdamas ant jo nuorodos.
					Sistema atidaro naują langą su informacija apie atitinkamą viešą renginį.
					Išsirinkęs renginį, vartotojas spaudžia mygtuką ,,Dalyvauti renginyje''. 
					Vartotojas įveda informaciją apie save (Vardas, Pavardė). 
					Sistema atidaro naują langą su užrašu ,,Registracija sėminga. Gero žaidimo''.
					
					\underline{Alternatyvūs scenarijai:}
					\begin{itemize}
						\item Jeigu vartotojo informacija yra netinkama, sistema atveria langą su prašymu įvesti informaciją iš naujo. 
						\item Jeigu pasirinktame renginyje vietos nėra, 
						sistema atidaro naują langą su užrašu: ,,Atsiprašome, vietų į šį renginį nėra. Pasirinkite kitą renginį.''
						\item Jeigu sistemoje nėra renginių, vartotojui pateikiama žinutė:
						,,Atsiprašome, jums prieinamų renginių nėra. Norėdami pamatyti daugiau pasirinkimų, prisijunkite.''
					\end{itemize}

				\begin{figure}[H]
					\centering
					\includegraphics[width=\textwidth]{img/MLP.png}
					\caption{Prisijungimo prie viešo renginio langas}
					\label{fig:prisijungti-prie-vieso-renginio}
				\end{figure}

				\item \textbf{Patikrinti renginio organizatoriaus informaciją} \\
					Vartotojas spaudžia mygtuką ,,Renginiai''. 
					Vartotojas pasirenka konkretų renginį. 
					Sistema atidaro naują langą su informacija apie atitinkamą renginį. 
					Vartotojas spaudžia ant teksto su užrašu ,,Organizatorius : [Vardas] [Pavardė]''.
					Sistema parodo detalesnę informaciją apie organizatorių.
					
					\underline{Alternatyvūs scenarijai:}
					\begin{itemize}
						\item Jeigu detalesnės informacijos apie organizatorių nėra, sistemoje pateikiamas užrašas ,,Šios inforamacijos nėra''.
						\item Jeigu sistemoje nėra renginių, vartotojui pateikiama žinutė:
						,,Atsiprašome, jums prieinamų renginių nėra. Norėdami pamatyti daugiau pasirinkimų, prisijunkite.''
					\end{itemize}

				\begin{figure}[H]
					\centering
					\includegraphics[width=\textwidth]{img/MLP.png}
					\caption{Renginio organizatoriaus informacijos patikrinimo langas}
					\label{fig:patikrinti-renginio-organizatoriaus-informacija}
				\end{figure}
			
			\item \textbf{Matyti privačius renginius} \\
					Vartotojas spaudžia mygtuką ,,Renginiai''.
					Atsidaro pasirinkimo langas, kuriame yra variantas ,,Renginių sąrašas'', kurį vartotojas pasirenka.
					Sistema atidaro visų vykstančių ir vyksiančių renginių sąrašą.
					Vartotojas filtravimo skirtyje ,,Rodyti:'' pasirenka variantą ,,Privačius renginius''.
					Renginių sąraše lieka tik privatūs renginiai.
					Sistema rodo 15 renginių viename puslapyje.
					
					\underline{Alternatyvūs scenarijai:}
					\begin{itemize}
						\item Jei sistemoje nėra jokių renginių, vietoj renginių sąrašo rodomas užrašas ,,Renginių nėra''.
						\item Jei sistemoje privačių renginių nėra, vietoj renginių sąrašo rodomas užrašas ,,Privačių renginių nėra''.
						\item Jei renginių yra 15 ar mažiau, puslapių numeriai nerodomi, nes visas sąrašas telpa viename puslapyje.
					\end{itemize}
				
				\begin{figure}[H]
					\centering
					\includegraphics[width=\textwidth]{img/MLP.png}
					\caption{Privačių renginių sąrašo langas}
					\label{fig:matyti-privacius-renginius}
				\end{figure}
			\item \textbf{Prašyti leidimo dalyvauti privačiame renginyje} \\
					Vartotojas atsidaro privačių renginių sąrašą (žr. užduotį ,,Matyti privačius renginius'').
					Atsidariusiame privačių renginių sąraše vartotojas pasirenka konkretų privatų renginį ir paspaudžia ant jo pavadinimo.
					Atsidariusiame konkretaus privataus renginio aprašyme vartotojas spaudžia mygtuką ,,Dalyvauti''.
					Iššoka langas, indikuojantis, kad užklausą pavyko sėkmingai nusiųsti organizatoriui.
					Organizatoriui patvirtinus prašymą dalyvauti, vartotojas gauna pranešimą apie galimybę dalyvauti renginyje.
					
					\underline{Alternatyvūs scenarijai:}
					\begin{itemize}
						\item Jei įvyksta klaida siunčiant užklausą organizatoriui, vartotojui parodomas langas su klaidos aprašymu ir prašymu pabandyti vėliau.
						\item Jei organizatorius nepatvirtina vartotojo dalyvavimo renginyje, sistema išsiunčia vartotojui pranešimą apie atmestą prašymą dalyvauti.
					\end{itemize}

				\begin{figure}[H]
					\centering
					\includegraphics[width=\textwidth]{img/MLP.png}
					\caption{Prašymo dalyvauti privačiame renginyje langas}
					\label{fig:prasymas-dalyvauti-privaciame-renginyje}
				\end{figure}
			\item \textbf{Peržiūrėti visų renginių tvarkaraštį} \\
				Vartotojas spaudžia mygtuką ,,Renginiai''.
				Atsidaro pasirinkimo langas, kuriame yra variantas ,,Renginių tvarkaraštis'', kurį vartotojas pasirenka.
				Sistema atidaro visų renginių tvarkaraštį, kuris vaizduojamas kalendoriaus forma.
				Kalendorius iš pradžių rodo einamąją dieną.
				Kalendorius vaizduoja vieno mėnesio informaciją viename lape.
			
				\underline{Alternatyvūs scenarijai:}
				\begin{itemize}
					\item Jei vartotojo įrenginys nepalaiko technologijos, reikalingos kalendoriaus atvaizdavimui, parodomas pranešimas ,,Jūsų įrenginys nepalaiko kalendoriaus funkcijos''.
					\item Jei renginių nėra, vaizduojamas tuščias kalendorius.
					\item Jei renginys trunka ne vieną dieną, jis vaizduojamas per kelis kalendoriaus langelius (dienas).
				\end{itemize}

				\begin{figure}[H]
					\centering
					\includegraphics[width=\textwidth]{img/MLP.png}
					\caption{Visų renginių tvarkaraščio langas}
					\label{fig:perziureti-renginiu-tvarkarasti}
				\end{figure}
			\item \textbf{Rašyti atsiliepimą apie renginį} \\
					Vartotojas spaudžia mygtuką ,,Renginiai''.
					Atsidaro pasirinkimo langas, kuriame yra variantas ,,Renginių sąrašas'', kurį vartotojas pasirenka.
					Vartotojas iš sąrašo pasirenka konkretų renginį ir spaudžia ant jo pavadinimo.
					Atsidariusiame konkretaus renginio aprašyme vartotojas spaudžia mygtuką ,,Rašyti atsiliepimą''.
					Atsidariusiame lange vartotojas pasirenka renginio įvertinimą (nuo 1 iki 10 balų) ir tam skirtame tekstiniame lange įrašo savo nuomonę apie renginį.
					Kai vartotojas baigia pildyti atsiliepimo laukus, jis spaudžia mygtuka ,,Siųsti''.
					Vartotojo atsiliepimas atsiduria bendrame renginio atsiliepimų sąraše.

				\underline{Alternatyvūs scenarijai:}
				\begin{itemize}
					\item Jei renginiui neleidžiama rašyti atsiliepimo, mygtuko ,,Rašyti atsiliepimą'' nėra.
					\item Jei vartotojas neužpildo visų atsiliepimo laukų, mygtukas ,,Siųsti'' yra neaktyvus.
					\item Jei įvyksta klaida užregistruojant vartotojo atsiliepimą, vartotojas apie tai informuojamas iššokančiame lange ir atsiliepimo forma išvaloma.
				\end{itemize}

				\begin{figure}[H]
					\centering
					\includegraphics[width=\textwidth]{img/MLP.png}
					\caption{Atsiliepimo apie renginį langas}
					\label{fig:rasyti-atsiliepima-apie-rengini}
				\end{figure}
			\item \textbf{Peržiūrėti informaciją apie buvusius renginius} \\
				Vartotojas spaudžia mygtuką ,,Renginiai''.
				Atsidaro pasirinkimo langas, kuriame yra variantas ,,Renginių sąrašas'', kurį vartotojas pasirenka.
				Vartotojas filtravimo skiltyje ,,Rodyti:'' pasirenka variantą ,,jau įvykusius renginius''.
				Renginių sąraše lieka tik jau vykę renginiai.
				Sistema rodo 15 renginių viename puslapyje.

				\underline{Alternatyvūs scenarijai:}
				\begin{itemize}
					\item Jeigu įvykusių renginių nėra, vietoj renginių sąrašo rodomas užrašas ,,Jau įvykusių renginių nėra''.
					\item Jeigu jau įvykusių renginių yra 15 ar mažiau, puslapių numeriai nerodomi, nes visi renginiai telpa viename puslapyje.
				\end{itemize}

				\begin{figure}[H]
					\centering
					\includegraphics[width=\textwidth]{img/MLP.png}
					\caption{Jau įvykusių renginių sąrašo langas}
					\label{fig:matyti-jau-ivykusius-renginius}
				\end{figure}
			
			\item \textbf{Paskelbti apie naują renginį} \\
				Organizatorius spaudžia mygtuką ,,Sukurti renginį''. 
				Organizatorius užpildo visą reikiamą informaciją apie renginį.
				Organizatorius spaudžia mygtuką ,,Paskelbti renginį''.
				
				\underline{Alternatyvūs scenarijai:}
				\begin{itemize}
					\item Jeigu informacija yra neteisinga arba nepilnai užpildyta, sistema neleidžia paspausti mygtuko ,,Paskelbti renginį''.
				\end{itemize}
				
				\begin{figure}[H]
					\centering
					\includegraphics[width=\textwidth, height=17cm, keepaspectratio]{img/MLP.png}
					\caption{Naujo renginio langas}
					\label{fig:paskelbti-nauja-rengini}
				\end{figure}
				\newpage
			\item \textbf{Paskirti naujo renginio atsakingą asmenį} \\
				Organizatorius spaudžia mygtuką ,,Mano renginiai''. 
				Organizatorius pasirenka redaguoti norimą renginį.
				Organizatorius spaudžia mygtuką ,,Paskirti atsakingą asmenį''. 
				Organizatorius įveda atsakingo asmens duomenis (vartotojo vardą).
				Organizatorius spaudžia mygtuką ,,Patvirtinti''.
				Sistema parodo užrašą ,,Sėkmingai priskirtas asmuo''.
				
				\underline{Alternatyvūs scenarijai:}
				\begin{itemize}
					\item Jeigu organizatorius įvedė neteisingus atsakingo asmens duomenis, sistema neleidžia paspausti mygtuko ,,Paskirti atsakingą asmenį''.
					\item Jeigu atsakingas asmuo jau yra prisikirtas, iššoka langas, klausiantis ar norima pakeisti atsakingo asmens duomenis.
				\end{itemize}
				
				\begin{figure}[H]
					\centering
					\includegraphics[width=\textwidth]{img/MLP.png}
					\caption{Atsakingo asmens paskyrimo langas}
					\label{fig:paskirti-atsakinga-asmeni}
				\end{figure}
				
			\item \textbf{Prie renginio pridėti aprašą ir nuotrauką} \\
				Organizatorius spaudžia mygtuką ,,Mano renginiai''. 
				Organizatorius pasirenka norimą redaguoti renginį.
				Organizatorius spaudžia mygtuką ,,Pridėti nuotrauką arba informaciją''. 
				Organizatorius prideda nuotrauką arba parašo naują informaciją apie renginį.
				Organizatorius spaudžia mygtuką ,,Atnaujinti''. 
				Sistema parašo pranešimą ,,Atnaujinta sėkmingai''.
				
				\underline{Alternatyvūs scenarijai:}
				\begin{itemize}
					\item Jeigu nuotrauka buvo įdėta neatitinkanti standartų, sistema jos nededa prie aprašymo ir išsiunčia pranešimą apie nuotraukos klaidą.
					\item Jeigu renginys jau yra pasibaigęs, sistema neleidžia pakeisti nuotraukos ar pakeisti aprašymo, bet įgalima papildyti jau esamą aprašymą.
				\end{itemize}
				
				\begin{figure}[H]
					\centering
					\includegraphics[width=\textwidth, height=8cm, keepaspectratio]{img/MLP.png}
					\caption{Renginio papildymo langas}
					\label{fig:prideti-aprasa}
				\end{figure}
				
			\item \textbf{Ištrinti renginį} \\
				Organizatorius spaudžia mygtuką ,,Mano renginiai''. 
				Organizatorius pasirenka norimą renginį.
				Organizatorius spaudžia mygtuką ,,Ištrinti renginį''. 
				Sistema parašo pranešimą su užrašu ,,Renginys ištrintas''. 
				Sistema ištrina atitinkamą renginį iš renginių sąrašo.
				Sistema išsiunčia pranešimą kiekvienam vartotojui, kuris yra prisijungęs prie to renginio, apie pašalintą renginį.
				
				\underline{Alternatyvūs scenarijai:}
				\begin{itemize}
					\item Jeigu nėra galimybės vartotojui išsiųsti pranešimą, sistema pranešimo nesiunčia.
					\item Jeigu renginys jau yra pasibaigęs, jo ištrinti sistema neleidžia, jis lieka sistemoje tarp jau vykusių renginių.
				\end{itemize}
				
				\begin{figure}[H]
					\centering
					\includegraphics[width=\textwidth, height=8cm, keepaspectratio]{img/MLP.png}
					\caption{Renginio ištrynimas}
					\label{fig:istrinti-rengini}
				\end{figure}
				
			\item \textbf{Priimti ar atmesti dalyvavimo renginyje prašymą} \\
				Organizatorius spaudžia mygtuką ,,Norintys dalyvauti renginyje''.
				Organizatorius pasirenka asmenį, kurį jis nori patikrinti. 
				Organizotorius spaudžia mygtuką ,,Patvirtinti dalyvavimą'' arba ,,Atmesti dalyvavimą''. 
				Sistema vartotojui atsiunčia pranešimą, su organizatoriaus pasirinktu atsakymu.
				
				\underline{Alternatyvūs scenarijai:}
				\begin{itemize}
					\item Jeigu nėra prašančiųjų dalyvauti, sistemoje iššoka langas sakantis organizatoriui, jog nėra naujų patvirtinimų.
					\item Jeigu dalyvis yra priimamas į renginį, bet nebėra vietų dalyvauti, jam išsiunčiama žinutė, jog jei atsiras laisvų vietų, jis bus priimtas į renginį.
				\end{itemize}
				
				\begin{figure}[H]
					\centering
					\includegraphics[width=\textwidth]{img/MLP.png}
					\caption{Prašymo dalyvauti peržiūra}
					\label{fig:priimti-dalyvi}
				\end{figure}

			\item \textbf{Priimti ar atmesti organizatoriaus paskyrimą} \\
				Atsakingas asmuo spaudžia mygtuką ,,Pasiūlymai''. 
				Atsakingas asmuo pasirenką norimą pasiūlymą.
				Atsakingas asmuo patvirtintina organizatoriaus paskyrimą būti atsakingam už konkretų renginį. 
				Sistema išsiunčia pranešimą organizatoriui, kad atsakingas asmuo priėmė paskyrimą.
				
				\underline{Alternatyvūs scenarijai:}
				\begin{itemize}
					\item Jeigu atsakingas asmuo atmeta organizatoriaus paskyrimą būti atsakingam už konkretų renginį, sistema išsiunčia pranešimą organizatoriui, kad atsakingas asmuo atmetė pranešimą.
					\item Jeigu asmuo priima pasiūlymą būti atsakingu asmeniu, bet atsakingu už renginį jau buvo paskirtas kitas žmogus, asmuo gauna pranešimą, kad jis taps atsakingu asmeniu, jei atsilaisvins vieta.
				\end{itemize}

				\begin{figure}[H]
					\centering
					\includegraphics[width=\textwidth]{img/MLP.png}
					\caption{Atsakingo asmens pasiūlymų valdymo langas}
					\label{fig:pasiulymu-sarasas}
				\end{figure}
			\item \textbf{Atsakyti į dalyvių klausimus}   \\
					Atsakingas asmuo spaudžia mygtuką ,,Mano renginiai''. 
					Atsakingas asmuo pasirenka norimą renginį.
					Atsakingas asmuo pasirenka klausimą į kurį nori atsakyti. 
					Sistema atidaro tekstinį langą, į kurį atsakingas asmuo įrašo atsakymą. 
					Atsakingas asmuo spaudžia mygtuką ,,Atsakyti''.
					Sistema siunčia atsakymą vartotojui, kuris jį uždavė.
					
					\underline{Alternatyvūs scenarijai:}
					\begin{itemize}
						\item Jeigu atsakingas asmuo nieko neparašo į tekstinį langą, sistema neleidžia paspausti mygtuko ,,Atsakyti''.
						\item Jeigu nėra užduota jokių naujų klausimų, sistemoje iššoka langas, pranešantis atsakingam asmeniui šią informaciją.
					\end{itemize}
				
				\begin{figure}[H]
					\centering
					\includegraphics[width=\textwidth, height=8cm, keepaspectratio]{img/MLP.png}
					\caption{Dalyvių klausimų atsakymo langas}
					\label{fig:atsakyti-klausimus}
				\end{figure}

			\subsection* {Reikalavimų - užduočių atsekamumo matrica}
			\begin{table}[H]
				\centering
				\caption{Reikalavimų - užduočių atsekamumo matrica}
				\label{ReikalavimuUzduociuAtsekamumoMatrica}
				\begin{tabular}{|
				>{\columncolor[HTML]{9B9B9B}}l |l|l|l|l|l|l|l|l|}
					\hline
					X   & \cellcolor[HTML]{9B9B9B}FR1 & \cellcolor[HTML]{9B9B9B}FR2 & \cellcolor[HTML]{9B9B9B}FR3 & \cellcolor[HTML]{9B9B9B}FR4 & \cellcolor[HTML]{9B9B9B}FR5 & \cellcolor[HTML]{9B9B9B}FR6 & \cellcolor[HTML]{9B9B9B}FR7 & \cellcolor[HTML]{9B9B9B}FR8 \\ \hline
					U1  &                             &                             & X                           &                             &                             &                             &                             &                             \\ \hline
					U2  &                             &                             & X                           &                             &                             &                             &                             &                             \\ \hline
					U3  &                             & X                           &                             & X                           & X                           & X                           &                             & X                           \\ \hline
					U4  &                             & X                           &                             & X                           &                             &                             &                             &                             \\ \hline
					U5  & X                           & X                           &                             & X                           & X                           &                             & X                           &                             \\ \hline
					U6  & X                           & X                           &                             & X                           &                             &                             &                             & X                           \\ \hline
					U7  &                             & X                           &                             & X                           &                             &                             &                             &                             \\ \hline
					U8  & X                           &                             &                             & X                           &                             &                             &                             &                             \\ \hline
					U9  & X                           &                             &                             & X                           &                             &                             &                             & X                           \\ \hline
					U10 &                             &                             &                             & X                           &                             &                             &                             &                             \\ \hline
					U11 &                             &                             &                             & X                           &                             &                             &                             &                             \\ \hline
					U12 &                             &                             &                             & X                           & X                           & X                           &                             &                             \\ \hline
					U13 &                             &                             &                             & X                           &                             &                             & X                           &                             \\ \hline
					U14 &                             &                             &                             &                             &                             &                             & X                           &                             \\ \hline
					U15 &                             &                             &                             & X                           &                             &                             &                             &                             \\ \hline
					U16 &                             &                             &                             & X                           &                             &                             &                             &                             \\ \hline
					U17 &                             &                             &                             & X                           &                             &                             &                             &                             \\ \hline
					U18 &                             &                             &                             & X                           &                             &                             & X                           &                             \\ \hline
					U19 &                             &                             &                             & X                           &                             &                             & X                           &                             \\ \hline
				\end{tabular}
			\end{table}
		\end{enumerate}
		
		
    \section{Peržiūros metu rastos klaidos} \label{klaidos}
		\begin{itemize}
			\item Dalykinės srities modelis suabstraktintas, nes iš pradžių buvo per daug lįsta į technines detales.
			\item FR2.4 Patikinti renginio organizatoriaus informaciją. ,,Patikinti'' pataisyta į ,,patikrinti''.
			\item Užduotis ,,Matyti privačius renginius''. Alternatyvus scenarijus, leidžiantis atfiltruoti tik privačius renginius perkeltas prie pagrindinio scenarijaus.
			\item Užduotis ,,Matyti privačius renginius''. Perrašyti alternatyvūs scenarijai.
			\item Pataisytos ir papildytos prisijungusio paprasto vartotojo užduotys.
			\item Pataisytos neprisijungusio vartotojo užduotys.Papildyti pagrindiniai ir alternatyvūs scenarijai.
			\item Sulietuvintos kabutės.
			\item 6pav. - ,,Preliuminarus'' pakeista į ,,Preliminarus''.
			\item Ištaisytos gramatikos ir skyrybos klaidos.
			
		\end{itemize}
    \section{Priedai}\label{priedai}
        \subsection{Pradiniai užsakovo reikalavimai sistemai} \label{priedai_uzsakovoReikalavimai}
		\begin{enumerate}
			\item Internetinis puslapis turi būti pasiekiamas ir neregistruotiem vartotojam.
			\item Neregistruotam vartotojui užėjus į puslapį, jam turi būti pateikiama tik neregistruotiem vartotojam skirta informacija (t.y. naujienos, renginių kalendorius, vaizdo įrašai ir pan.)
			\item Būtina prisijungti norint pirkti bilietus, registruotis į renginį kaip dalyviui, aplikuoti į siūlomas darbo pozicijas.
			\item Kiekvienas vartotojas privalo turėti vartotojo vardą bei slaptažodį.
			\item Kiekvienas vartotojas gali anketoje suvesti papildomą asmeninę informaciją - vardą, pavardę, gimimo datą, telefono numerį, gyvenamąją vietą ir pan.
			\item Registruojantis į renginį ar aplikuojant į darbo pozicijas, vartotojo anketoje papildoma informacija yra privaloma.
			\item Registruojantis į komandinį renginį, reikia pasirinkti komandą, su kuria bus dalyvaujama. Jei komanda dar nesukurta, ją reikia sukurti registacijos metu anketoje.
			\item Registracijos metu anketoje kuriant komandą, reikia nurodyti komandos pavadinimą bei žmones, kurie bus komandos nariai.
			\item Pridėtiems komandos nariams yra išsiunčiami kvietimai, nariai gali juos priimti ar atmesti.
			\item Vaizdo įrašų numatytoji rikiavimo tvarka turi būti pagal datą, tačiau ją galima keisti (pvz. pagal peržiūras ar pan.)
			\item Rezultatų lentelės pateikiamos prie atitinkamo renginio.
			\item Dalyviai rezultatų lentelėje rikiuojami pagal jų rezultatą.
			\item Rezultatų lentelę galima filtruoti pagal kiekvieną atributą.
			\item Turi būti pateikiama sporto šakų, dalyvių bei komandų rezultatų istorija. Gali būti pateikiama tinkama statistika.
			\item Internetinėje svetainėje yra visiems prieinama sritis "Pateikti pasiūlymą", kurioje kiekvienas lankytojas gali pateikti pasiūlymą renginio organizatoriams.
			\item Pateikiant pasiūlymą reikia nurodyti savo elektroninį paštą (jei vartotojas prisijungęs - šis laukas užpildomas automatiškai) bei trumpą idėjos aprašą.
			\item Rėmėjų logotipai atvaizduojami internetiniame puslapyje tam skirtose vietose.
			\item Prisijungus kaip administratoriui, turi atsirasti prieiga prie administratoriaus skilties.
			\item Administratoriaus skiltyje turi būti galimybė peržiūrėti, pridėti, ištrinti bei atnaujinti naujienas, renginius, renginio rezultatus bei vaizdo įrašus vaizdo įrašus, apriboti ar blokuoti vartotojo prieigą, matyti visų dalyvių, renginių, aplikaciją pateikusių potencialių darbuotojų sąrašus.
			\item Prie kiekvieno renginio esančioje bilietų skiltyje, turi būti nurodomas bilietų likutis ir vieno bilieto kaina.
			\item Internetiniame puslapyje turi būti galimybė pasirinkti kitą kalbą iš lietuvių, anglų, rusų.
			\item Internetinis puslapis turi turėti savo analogą - mobiliąją programėlę. Visas funkcionalumas, esantis internetiniame puslapyje, turi būti ir mobiliojoje programėlėje. (Optional)
		\end{enumerate}
	\subsection{Užsakovo reikalavimų pakeitimai} \label{priedai_uzsakovoReikalavimaiPakeitimai}
		\begin{itemize}
			\item Visi reikalavimai susisteminti ir labiau klarifikuoti.
			\item 4 ir 5 reikalavimai sujungti į vieną.
			\item Dėl 2, 17, 19, 21 bei 22 reikalavimų buvo susisiekta su užsakovais. Pokalbiai pateikti žemiau.
			\item Reikalavimas numeris 17:
				\begin{itemize}
					\item \textit{Mes:} Sveiki, rašome jums iš UAB ,,Festofilas software group LT'', norime iš jūsų gauti patikslinimą dėl reikalavimo numeris 17. Norėtume tiksliau sužinoti vietas, kuriuose bus pateikti rėmėjų logotipai. Iš anksto ačiū.
					\item \textit{Užsakovas NR1:} ????
					\item \textit{Užsakovas NR1:} Su darbo vafovu
					\item \textit{Mes:} Ačiū už informaciją.
					
					\item \textit{Mes:} Sveiki, rašome jums iš UAB ,,Festofilas software group LT'', norime iš jūsų gauti patikslinimą dėl reikalavimo numeris 17. Norėtume tiksliau sužinoti vietas, kuriuose bus pateikti rėmėjų logotipai. Iš anksto ačiū.
					\item \textit{Užsakovas NR2:} Šias. {PlaceHolder for small smiley face}
					\textbf{Po 18 minučių}
					\item \textit{Užsakovas NR2:} Patikslinimas: Rėmėjų logotipams skirtos vietos yra: Puslapio viršuje, kairėje, dešinėje ir apačioje
					\item \textit{Užsakovas NR2:} :S
					\item \textit{Užsakovas NR2:} {PlaceHolder for big smiley face}
					\item \textit{Mes:} Ačiū už informaciją.
				\end{itemize}
			\item Reikalavimai numeris 21, 22:
				\begin{itemize}
					\item \textit{Mes:} Sveiki, rašome jums iš UAB ,,Festofilas software group LT'', norime pranešti, kad dėl laiko ir lėšų stokos mes turime atsisakyti, jūsų pateikto reikalavimo numeris 22, bei apriboti reikalavimą numeris 21 ties anglų kalba. Ar jums tinka?
					\item \textit{Užsakovas NR3:} Užsakovai nesupyks, jei atliksite viską vėluodami viena savaite!
					\item \textit{Mes:} Supratome, padarysime atsižvelgiant į esamus resursus.
					\item \textit{Užsakovas NR3:} Puiku, lauksime darbo pristatymo šį antradienį. Nors mūsų peržiūroje ir nebus.
				\end{itemize}
			\item Reikalavimai numeris 2, 19:
				\begin{itemize}
					\item \textit{Mes:} Sveiki, rašome jums iš UAB ,,Festofilas software group LT'', norime iš jūsų gauti patikslinimą dėl reikalavimų numeris 2 ir 19. Norėtume sužinoti kad bus vaizduojama naujienų skiltyje. Iš anksto ačiū.
					\item \textit{Užsakovas NR2:} :DDDDDDDDDDD
					\item \textit{Užsakovas NR2:} jūs dėl kiekvieno rašysit? small smiley :D
					\item \textit{Mes:} Kolkas kituose reikalavimuose nematome diskutuotinų objektų, todėl nematome reikalo apie juos diskutuoti. Bet jei atsiras neaiškumų tai būtinai atsiklausime, norėdami užtikrinti mūsų paslaugų kokybę.
					\item \textit{Užsakovas NR2:} turėkit vaizduotės šiek tiek. small smiley winking tai gali būti artėjantis koks nors įdomesnis renginys, praėjusio renginio rezultatai ir pan.
					\item \textit{Užsakovas NR2:} tai kad jau gal dėl 5 gavome žinučių. small smiley smile
					\item \textit{Mes:} Primenu, kad išviso jūs pateikėte mums 22 reikalavimus
					\item \textit{Užsakovas NR2:} dėkoju už priminimą ir atsiprašau dėl nesklandumų. Tikiuosi daugiau jų nekils. Tačiau visada galite kreiptis.
					\item \textit{Mes:} Iškilo dar vienas klausimas: o kas ir kaip turėtų pateikti tas naujienas?
					\item \textit{Užsakovas NR2:} Organizatoriai pasirūpins jų įkėlimu. small smiley smile
					\item \textit{Mes:} Supratome, dėkojame už greitus atsakymus.
					\item \textit{Užsakovas NR2:} Visada malonu bendradarbiauti. Pagarbiai „LtNSO“
					
					\item \textit{Mes:} Sveiki, rašome jums iš UAB ,,Festofilas software group LT'', norime iš jūsų gauti patikslinimą dėl reikalavimų numeris 2 ir 19. Norėtume sužinoti kad bus vaizduojama naujienų skiltyje. Iš anksto ačiū.
					\item \textit{Užsakovas NR1:} Naujienos apie ateinancius ar buvusius renginiua
					\item \textit{Užsakovas NR1:} Arba tiesiogine "live" apzvalga is pacio renginio
					\item \textit{Mes:} Visvien neaiškų kas ir kaip turėtų pateikti tas naujienas.
					\item \textit{Užsakovas NR1:} Tai adminai arba organizatoriai renginio0
				\end{itemize}
        \end{itemize}
    \sectionnonum{Literatūros sąrašas} \label{literaturosSarasas}
        \begin{itemize}
			\item Doc. dr. K. Petrausko Programų Sistemų Inžinerijos kurso konspektai
			\item Doc. dr. K. Petrausko Pirmojo laboratorinio darbo struktūra iš: http://www.mif.vu.lt/~karolis/PSI2.html
			\item A. Abran, J. W. Moore, P.Bourque, R. Dupuis, L. L. Tripp - ,,Guide to the Software Engineering Body of Knowledge''
			\item Latex dokumentacija: http://www.latex-project.org/help/documentation/
        \end{itemize}
\end{document}