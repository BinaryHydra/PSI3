\documentclass{VUMIFPSkursinis}
\usepackage{algorithmicx}
\usepackage{algorithm}
\usepackage{algpseudocode}
\usepackage{amsfonts}
\usepackage{amsmath}
\usepackage{bm}
\usepackage{caption}
\usepackage{color}
\usepackage{float}
\usepackage{graphicx}
\usepackage{listings}
\usepackage{subfig}
\usepackage{wrapfig}

% Titulinio aprašas
\university{Vilniaus universitetas}
\faculty{Matematikos ir informatikos fakultetas}
\department{Programų sistemų katedra}
\papertype{Kursinis darbas}
\title{Programų sistemų kūrimo metodų tyrimas}
\titleineng{Investigation Methods of Software Development}
\status{3 kurso ... grupės studentas}
\author{Vardenis Pavardenis}
% \secondauthor{Vardonis Pavardonis}   % Pridėti antrą autorių
% \thirdauthor{Vardonis Pavardonis}   % Pridėti trečią autorių
% \fourthauthor{Vardonis Pavardonis}   % Pridėti ketvirtą autorių
\supervisor{prof. habil. dr. Vardaitis Pavardaitis}
\date{Vilnius – \the\year}

% Nustatymai
% \setmainfont{Palemonas}   % Pakeisti teksto šriftą į Palemonas (turi būti įdiegtas sistemoje)
\bibliography{bibliografija}

\begin{document}
\maketitle

\tableofcontents

\sectionnonum{Įvadas}
Įvade apibūdinamas darbo tikslas, temos aktualumas ir siekiami rezultatai.
Darbo įvadas neturi būti dėstymo santrauka. Įvado apimtis 1–2 puslapiai.

\section{Medžiagos darbo tema dėstymo skyriai}
Medžiagos darbo tema dėstymo skyriuose pateikiamos nagrinėjamos temos detalės:
pradinė medžiaga, jos analizės ir apdorojimo metodai, sprendimų įgyvendinimas,
gautų rezultatų apibendrinimas. Šios dalies turinys labai priklauso nuo darbo
temos. Skyriai gali turėti poskyrius ir smulkesnes sudėtines dalis, kaip
punktus ir papunkčius.

Medžiaga turi būti dėstoma aiškiai, pateikiant argumentus. Tekstas dėstomas
trečiuoju asmeniu, t.y. rašoma ne „aš manau“, bet „autorius mano“, „autoriaus
nuomone“. Reikėtų vengti informacijos nesuteikiančių frazių, pvz., „...kaip jau
buvo minėta...“, „...kaip visiems žinoma...“ ir pan., vengti grožinės literatūros
ar publicistinio stiliaus, gausių metaforų ar panašių meninės išraiškos
priemonių.

\subsection{Poskyris}
Citavimo pavyzdžiai: cituojamas vienas šaltinis \cite{PvzStraipsnLt}; cituojami
keli šaltiniai \cite{PvzStraipsnEn, PvzKonfLt, PvzKonfEn, PvzKnygLt, PvzKnygEn,
PvzElPubLt, PvzElPubEn, PvzMagistrLt, PvzPhdEn}.

\subsubsection{Skirsnis}
\subsubsubsection{Straipsnis}
\subsubsection{Skirsnis}
\section{Skyrius}
\subsection{Poskyris}
\subsection{Poskyris}

\sectionnonum{Rezultatai ir išvados}
Rezultatų ir išvadų dalyje turi būti aiškiai išdėstomi pagrindiniai darbo
rezultatai (kažkas išanalizuota, kažkas sukurta, kažkas įdiegta) ir pateikiamos
išvados (daromi nagrinėtų problemų sprendimo metodų palyginimai, teikiamos
rekomendacijos, akcentuojamos naujovės).

\printbibliography[heading=bibintoc]  % Šaltinių sąraše nurodoma panaudota
% literatūra, kitokie šaltiniai. Abėcėlės tvarka išdėstomi darbe panaudotų
% (cituotų, perfrazuotų ar bent paminėtų) mokslo leidinių, kitokių publikacijų
% bibliografiniai aprašai.  Šaltinių sąrašas spausdinamas iš naujo puslapio.
% Aprašai pateikiami netransliteruoti. Šaltinių sąraše negali būti tokių
% šaltinių, kurie nebuvo paminėti tekste.

% \sectionnonum{Sąvokų apibrėžimai}
\sectionnonum{Santrumpos}
Sąvokų apibrėžimai ir santrumpų sąrašas sudaromas tada, kai darbo tekste
vartojami specialūs paaiškinimo reikalaujantys terminai ir rečiau sutinkamos
santrumpos.

\appendix  % Priedai
% Prieduose gali būti pateikiama pagalbinė, ypač darbo autoriaus savarankiškai
% parengta, medžiaga. Savarankiški priedai gali būti pateikiami ir
% kompaktiniame diske. Priedai taip pat numeruojami ir vadinami. Darbo tekstas
% su priedais susiejamas nuorodomis.

\section{Niauroninio tinklo struktūra}
\begin{figure}[H]
    \centering
    \includegraphics[scale=0.5]{img/MLP}
    \caption{Paveikslėlio pavyzdys}
    \label{img:mlp}
\end{figure}


\section{Eksperimentinio palyginimo rezultatai}
% tablesgenerator.com - converts calculators (e.g. excel) tables to LaTeX
\begin{table}[H]\footnotesize
  \centering
  \caption{Lentelės pavyzdys}
  {\begin{tabular}{|l|c|c|} \hline
    Algoritmas & $\bar{x}$ & $\sigma^{2}$ \\
    \hline
    Algoritmas A  & 1.6335    & 0.5584       \\
    Algoritmas B  & 1.7395    & 0.5647       \\
    \hline
  \end{tabular}}
  \label{tab:table example}
\end{table}

\end{document}
